\documentclass{article}

%{{{ prelude
\usepackage{bussproofs}
\usepackage[english]{babel}
\usepackage{amsmath}
\usepackage{amsfonts}
\usepackage{amssymb}
\usepackage{amsthm}
\usepackage{paralist}
\usepackage[colorlinks=false,pdfborder={0 0 0},
pdftitle={superposition modulo modular arithmetic},pdfauthor={Simon Cruanes}]{hyperref}

\newcommand{\set}[1]{\ensuremath{\text{set}({#1})}}

\title{Superposition and Chaining with Sets}
%}}}

\begin{document}

\maketitle

% XXX: on a un ensemble maximal (top) pour chaque type
%       ----> complémentaire ? peut être bien pratique !

\section{Basics}
%{{{

\subsection{Definitions}
%{{{
Literals are of the form $\bigcap_i a_i \subseteq \bigcup_j b_j$
or $\bigcup_i a_i \not\subseteq \bigcap_j b_j$
where the $a_i$ and $b_j$ are terms of type set.
We write
$\bigcap_i a_i \subseteq \emptyset$ if the set of $b_j$ is empty (since
$\emptyset$ is the neutral element for $\cup$).
The terms of type $\set{\alpha}$
are of the form:

\begin{itemize}
    \item $\{ t \}$ where $t : \alpha$;
    \item $ f(t_1,\dots,t_n) $ where $f$ is a function of domain set.
\end{itemize}
%}}}

\subsection{Powerset}
%{{{
\noindent{}We can encode the powerset operator $\mathbb{P}$ as follows, using a fresh
variable $s:\set{\alpha}$ and a fresh skolem
symbol $k:\set{\alpha}$ where $\mathbb{P}: \set{\set{\alpha}}$.
\begin{itemize}
\item If $\mathbb{P}(t) \cap A \subseteq B$,
    it means $\forall s, s\in \mathbb{P}(t) \cap A \Rightarrow s \in B$;
    $s \in \mathbb{P}(t) \cap A$ is synonymous to $s \subseteq t \land
    \{ s \} \subseteq A$.
\item Similarly, $A \subseteq \mathbb{P}(t) \cup B$
    means $\forall s, s\in A \Rightarrow s \in \mathbb{P}(t) \lor s\in B$,
    hence $\forall s, \{s \}\subseteq A \Rightarrow s \subseteq t \lor \{ s \} \subseteq B$.
\item $\mathbb{P}(t) \cup A \not\subseteq B$ means that there exists
    some set $s$ such that $s \subseteq t \lor \{s\} \in A$ but $\{s\} \not\in B$.
    Therefore we transform this into two clauses.
\item $A \not\subseteq \mathbb{P}(t) \cap B$ is similar, but $s$ not being
    a member of $\mathbb{P}(t)\cap B$ means that $s \not\in B \lor s\not\subseteq t$.
    Therefore we translate it into the two clauses obtained from
    $\exists s, \{s\} \subseteq A \land (\{s\}\not\subseteq B \lor s\not\subseteq t)$.
\end{itemize}

When $\mathbb{P}$ occurs in a clause, the rules to eliminate it are shown
in Figure~\ref{fig:powerset}.

\begin{figure}[htp]
%{{{
\begin{prooftree}
\AxiomC{$ C \lor \mathbb{P}(t) \cap A \subseteq B$}
\doubleLine
\UnaryInfC{$C \lor s \not\subseteq t \lor \{ s \} \not\subseteq A \lor \{ s \} \subseteq B$}
\end{prooftree}

\begin{prooftree}
\AxiomC{$ C \lor A \subseteq \mathbb{P}(t) \cup B$}
\doubleLine
\UnaryInfC{$C \lor \{ s \} \not\subseteq A \lor s \subseteq t \lor \{ s \} \subseteq B$}
\end{prooftree}

\begin{prooftree}
\AxiomC{$ C \lor \mathbb{P}(t) \cup A \not\subseteq B $}
\doubleLine
\UnaryInfC{$C \lor \{k\}\subseteq A \lor k \subseteq t 
    \qquad   C \lor \{k\}\not\subseteq B$}
\end{prooftree}

\begin{prooftree}
\AxiomC{$ C \lor A \not\subseteq \mathbb{P}(t) \cap B $}
\doubleLine
\UnaryInfC{$ C \lor k \subseteq A
    \qquad  C \lor \{k\} \not\subseteq B \lor k \not\subseteq t$}
\end{prooftree}
\caption{Elimination of $\mathbb{P}$}
\label{fig:powerset}
\end{figure}
%}}}

%}}}
%}}}

\section{Skolemisation and CNF}
%{{{
The numerous set operators must be translated to the rigid clause format
we defined above; Figure~\ref{fig:skolem} describes the translation process
that must take place after NNF and before Skolemization.
In the rules, $A$ and $B$ have type $\set{\alpha}$, $s$ is a term
of type $\alpha$, and $x:\alpha$ is a
variable. Then, usual CNF transformation, with skolemization
and distribution of $\land$ over $\lor$, can proceed as usual.

\begin{figure}[htbp]
%{{{
\begin{center}
\begin{align*}
% basics
A \not\subset B &\leadsto (B \subseteq A \lor A \not\subseteq B) \\
A \not= B       &\leadsto (A \not\subseteq B \lor B \not\subseteq A) \\
A \subset B     &\leadsto (A \subseteq B \land B \not\subseteq A) \\
A = B           &\leadsto (A \subseteq B \land B \subseteq A) \\
s \in A         &\leadsto \{ s \} \subseteq A \\
s \not\in A     &\leadsto \{ s \} \not\subseteq A \\
\text{nonempty}(A)          &\leadsto A \not\subseteq \emptyset \\
\lnot \text{nonempty}(A)    &\leadsto A \subseteq \emptyset \\
% subseteq (FIXME)
%A \subseteq B \cup (C \backslash D)
%    &\leadsto A \subseteq B \cup C \land A \cap D \subseteq B \\
%A \subseteq B \backslash C
%    &\leadsto A \subseteq B \land A \cap C \subseteq \emptyset \\
%A \cap (B \backslash D) \subseteq C
%    &\leadsto A \cap B \subseteq C \cup D \\
%A \backslash B \subseteq C
%    &\leadsto A \subseteq B \cup C \\
%A \cap (B \cup D) \subseteq C
%    & \leadsto A \cap B \subseteq C \land A \cap D \subseteq C \\
%A \cup B \subseteq C
%    & \leadsto A \subseteq C \land B \subseteq C \\
%A \subseteq B \cup (C \cap D)
%    & \leadsto A \subseteq B \cup C \land A \subseteq B \cup D \\
%A \subseteq B \cap C
%    & \leadsto A \subseteq B \land A \subseteq C \\
%% not subseteq
%A \not\subseteq B \cap (C \backslash D)
%    &\leadsto A \cup D \not\subseteq B \cap C \lor A \cap D \not\subseteq B
%    % exists x, x in A and x notin (B cap (C \ D)). so
%    % x notin B cap C, or x in D (so not in C\D)
%    \\
%A \not\subseteq B \backslash C
%    &\leadsto A \not\subseteq B \lor A \cap C \not\subseteq \emptyset \\
%A \cup (B \backslash D) \not\subseteq C
%    &\leadsto A \cap B \not\subseteq C \cup D \\
%A \backslash B \not\subseteq C
%    &\leadsto A \not\subseteq B \cup C \\
A \cap B \not\subseteq C
    & \leadsto A \not\subseteq C \land B \not\subseteq C \\
A \cup (B \cap D) \not\subseteq C
    & \leadsto A \cup B \not\subseteq C \land A \cup D \not\subseteq C \\
A \not\subseteq B \cup C
    & \leadsto A \not\subseteq B \land A \not\subseteq C \\
A \not\subseteq B \cap (C \cup D)
    & \leadsto A \not\subseteq B \cap C \land A \not\subseteq B \cap D \\
% empty set, singleton
A \cap \emptyset
    & \leadsto \emptyset \\
A \cup \emptyset
    & \leadsto A \\
\emptyset \subseteq B
    & \leadsto \top \\
\emptyset \not\subseteq B
    & \leadsto \bot \\
\{ s \} \subseteq \emptyset
    & \leadsto \bot \\
\{ s \} \not\subseteq \emptyset
    & \leadsto \top \\
\end{align*}
\caption{Skolemization and CNF rules}
\label{fig:skolem}
\end{center}
%}}}
\end{figure}

%}}}

\section{Inference Rules}
%{{{
We mix \emph{The Strive-based Subset Prover} and
\emph{Rewrite Techniques for Transitive Relations} papers for our
inference rules.
We will pervasively use monotonicity properties $A \subseteq B \Rightarrow A \cap C
\subseteq B \cap C$ and $A \subseteq B \Rightarrow A \cup C \subseteq B \cup C$.
The rules are exposed in Figure~\ref{fig:rules}. We will use capitalized
letter such as $A_\cap$ to denote $\bigcap_i a_i$, and $A_\cup$ to denote
$\bigcup_i a_i$.

All the inference operate on ground clauses, but can easily be lifted to
first-order terms by first unifying terms. Since $\cap$ and $\cup$ are
idempotent, several terms on one side of a literal can be unified at once
without trouble (unlike, for instance, unification of terms in arithmetic
sums or surrounded with AC operators). For instance positive chaining
for first-order clauses is (assuming the sets $I$ and $J$ are non empty,
none of the $t_i, t'_j$ are variables
and the two clauses do not share variables):

\begin{prooftree}
    \AxiomC{$ K \lor A_\cap \bigcap_{i\in I} t_i \subseteq B_\cup $}
    \AxiomC{$ K' \lor A'_\cap \subseteq B_\cup \cup \bigcup_{j\in J} t'_j $}
    \RightLabel{$\forall i,j,~ t_i\sigma = t'_j\sigma$}
    \BinaryInfC{$ (K \lor K' \lor A_\cap \cap A'_\cap
        \subseteq B_\cup \cup B'_\cup)\sigma $}
\end{prooftree}

Let us develop the chaining inferences. Say we have $A_\cap \subseteq t\cup B_\cup$
and $t \cap A'_\cap \subseteq B'_\cup$, then we can deduce intermediate literals
$A_\cap \cap A'_\cap \subseteq (t \cup B_\cup) \cap A'_\cap$ and
$(t\cap A'_\cap)\cup B_\cup \subseteq B'_\cup \cup B_\cup$. Then, we use the fact that
$(t \cup B_\cup) \cap A'_\cap = (t\cap A'_\cap) \cup (B_\cup\cap A'_\cap)$, and that
$B_\cup \cap A'_\cap \subseteq B_\cup$, so
$(t\cap A'_\cap) \cup (B_\cup\cap A'_\cap) \subseteq (t \cap A'_\cap)\cup B_\cup$.
Therefore by transitivity of $\subseteq$, we obtain
$A_\cap\cap A'_\cap \subseteq B_\cup \cup B'_\cup$.

% a & t => b
% a' & t not=> b'
%     ---> exists x, x in a' & x in t & x notin b'
%     ---> if b => b', then x in a & a' & t => x in b => x in b'  ---> absurd
%     ---> therefore   b not=> b' | (x in (a' & t) \ a)
%     ---> therefore,  b not=> b' | (a' & t) not=> a

Conversely, the negative chaining rules use the contrapositive of transitivity,
that is, $A \subseteq B \land A \not\subseteq C \Rightarrow B \not\subseteq C$
(otherwise we would have $A \subseteq B \subseteq C$ and therefore $A \subseteq
C$)\footnote{more precisely we use
$A\subseteq B \land A\subseteq A' \land A' \not\subseteq C \Rightarrow $
}.
Therefore, from $A_\cap \cap t \subseteq B_\cup$ and
$A'_\cup \cup t \not\subseteq B'_\cap$ we
deduce $(A_\cap\cap t)\cup A'_\cup \subseteq B_\cup\cup A'_\cup$ by monotonicity.


and $(A'_\cup\cup t) \cap A_\cap \not\subseteq B'_\cap\cap A_\cap$ by monotonicity.
Since $(A'_\cup \cup t)\cap A_\cap
    = (A'_\cup \cap A_\cap)\cup (A_\cap\cap t)
    \subseteq (A_\cap\cap t) \cup A'_\cup$.
Therefore $B_\cup\cup A'_\cup \not\subseteq B'_\cap\cap A_\cap$.
A similar reasoning applies when $t$ occurs at the right-hand side of
$\subseteq$ and $\not\subseteq$, using the transitivity rule
$A \subseteq B \land C \not\subseteq B \Rightarrow C \not\subseteq A$.

Factoring rules are used to \emph{merge} together two literals whose maximal
terms are the same. The basic idea is that if $A \subseteq B
\lor A \subseteq C$, then if we assume $B\subseteq C$ we can deduce
$A \subseteq B \Rightarrow A \subseteq C$. Therefore $A\subseteq C$ is true
in both cases and we can merge together the two literals provided $B\subseteq C$.

Say we have
$t \cap A_\cap \subseteq B_\cup \lor t \cap A'_\cap \subseteq B'_\cup$; if
TODO %TODO factoring

\begin{figure}[htp]
%{{{
\begin{center}

\begin{prooftree}
    \AxiomC{$ K\lor (A_\cap \subseteq t \cup B_\cup) $}
    \AxiomC{$ K'\lor (t \cap A'_\cap \subseteq B'_\cup) $}
    \RightLabel{positive chaining}
    \BinaryInfC{$ K \lor K' \lor (A_\cap \cap A'_\cap \subseteq B_\cup \cup B'_\cup) $}
\end{prooftree}

\begin{prooftree}
    \AxiomC{$ K\lor (A_\cap \cap t \subseteq B_\cup) $}
    \AxiomC{$ K'\lor (A'_\cup \cup t \not\subseteq B'_\cap) $}
    \RightLabel{negative chaining left}
    \BinaryInfC{$ K \lor K' \lor (A'_\cup \cup B_\cup \not\subseteq A_\cap \cap B'_\cap) $}
\end{prooftree}

\begin{prooftree}
    \AxiomC{$ K\lor (A_\cap \subseteq B_\cup \cup t) $}
    \AxiomC{$ K'\lor (A'_\cup \not\subseteq B'_\cap \cap t) $}
    \RightLabel{negative chaining right}
    \BinaryInfC{$ K \lor K' \lor (A'_\cup \cup B_\cup \not\subseteq A_\cap \cap B'_\cap) $}
\end{prooftree}

\begin{prooftree}
    \AxiomC{$ K
        \lor (t \cap A_\cap \subseteq B_\cup)
        \lor (t \cap A'_\cap \subseteq B'_\cup) $}
    \RightLabel{factoring left}
    \UnaryInfC{$ K \lor B_\cup \not\subseteq A'_\cap \lor TODO$} % TODO
\end{prooftree}

\mbox{
    \AxiomC{$ K \lor \{ t \} \subseteq \emptyset $}
    \doubleLine
    \UnaryInfC{$ K $}
    \DisplayProof

    and

    \AxiomC{$ K \lor \{ t \} \not\subseteq \emptyset $}
    \RightLabel{emptyset}
    \doubleLine
    \UnaryInfC{$~$}
    \DisplayProof
} \\[10pt]


\mbox{
    \AxiomC{$ K \lor t \cap A_\cap \subseteq t\cup B_\cup $}
    \doubleLine
    \UnaryInfC{$~ $}
    \DisplayProof

    and

    \AxiomC{$ K \lor t \cap A_\cap \not\subseteq t\cup B_\cup $}
    \RightLabel{reflexivity}
    \doubleLine
    \UnaryInfC{$K$}
    \DisplayProof
} \\[10pt]



\caption{Inference Rules}
\label{fig:rules}
\end{center}
%}}}
\end{figure}

%}}}


\end{document}
