\documentclass{article}

%{{{ prelude
\usepackage{bussproofs}
\usepackage[english]{babel}
\usepackage{amsmath}
\usepackage{amsfonts}
\usepackage{amssymb}
\usepackage{amsthm}
\usepackage{paralist}
\usepackage[colorlinks=false,pdfborder={0 0 0},
pdftitle={superposition modulo modular arithmetic},pdfauthor={Simon Cruanes}]{hyperref}

\newcommand{\set}[1]{\ensuremath{\text{set}({#1})}}

\title{Superposition and Chaining with Sets}
%}}}

\begin{document}

\maketitle

\section{Basics}
%{{{

\subsection{Definitions}
%{{{
Literals are of the form $\bigcap_i a_i \subseteq \bigcup_j b_j$
or $\bigcap_i a_i \not\subseteq \bigcup_j b_j$
where the $a_i$ and $b_j$ are terms of type set.
$\dot\subseteq$ stands for either $\subseteq$ or $\not\subseteq$.
We write
$\bigcap_i a_i ~\dot\subseteq~ \emptyset$ if the set of $b_j$ is empty (since
$\emptyset$ is the neutral element for $\cup$).
The terms of type $\set{\alpha}$
are of the form:

\begin{itemize}
    \item $\{ t \}$ where $t : \alpha$;
    \item $ f(t_1,\dots,t_n) $ where $f$ is a function of domain set.
\end{itemize}
%}}}

\subsection{Powerset}
%{{{
\noindent{}We can encode the powerset operator $\mathbb{P}$ as follows, using a fresh
variable $s:\set{\alpha}$ where $\mathbb{P}: \set{\set{\alpha}}$. If
$\mathbb{P}(t) \cap A \subseteq B$,
it means $\forall s, s\in \mathbb{P}(t) \cap A \Rightarrow s \in B$;
$s \in \mathbb{P}(t) \cap A$ is synonymous to $s \subseteq t \land
\{ s \} \subseteq A$. Conversely, $A \subseteq \mathbb{P}(t) \cup B$
means $\forall s, s\in A \Rightarrow s \in \mathbb{P}(t) \lor s\in B$,
hence $\forall s, \{s \}\subseteq A \Rightarrow s \subseteq t \lor \{ s \} \subseteq B$.

When $\mathbb{P}$ occurs in a clause, the rule is hence:

\begin{prooftree}
\AxiomC{$ C \lor \mathbb{P}(t) \cap A \subseteq B$}
\doubleLine
\UnaryInfC{$C \lor s \not\subseteq t \lor \{ s \} \not\subseteq A \lor \{ s \} \subseteq B$}
\end{prooftree}

and

\begin{prooftree}
\AxiomC{$ C \lor A \subseteq \mathbb{P}(t) \cup B$}
\doubleLine
\UnaryInfC{$C \lor \{ s \} \not\subseteq A \lor s \subseteq t \lor \{ s \} \subseteq B$}
\end{prooftree}
%}}}
%}}}

\section{Skolemisation and CNF}
%{{{
The numerous set operators must be translated to the rigid clause format
we defined above; Figure~\ref{fig:skolem} describes the translation process
that must take place after NNF and before Skolemization.
In the rules, $A$ and $B$ have type $\set{\alpha}$, $s$ is a term
of type $\alpha$, and $x:\alpha$ is a
variable. The rules for simplifying $\not\subseteq$ are obtained by
transforming $A \not\subseteq B$ into $\lnot(A \subseteq B)$, applying
adequate rules on $\subseteq$, and then putting $\lnot$ back and
applying NNF again. Then, usual CNF transformation, with skolemization
and distribution of $\land$ over $\lor$, can proceed.

\begin{figure}[htbp]
%{{{
\begin{center}
\begin{align*}
A \not\subset B &\leadsto (B \subseteq A \lor A \not\subseteq B) \\
A \not= B       &\leadsto (A \not\subseteq B \lor B \not\subseteq A) \\
A \subset B     &\leadsto (A \subseteq B \land B \not\subseteq A) \\
A = B           &\leadsto (A \subseteq B \land B \subseteq A) \\
s \in A         &\leadsto \{ s \} \subseteq A \\
s \not\in A     &\leadsto \{ s \} \not\subseteq A \\
\text{nonempty}(A)          &\leadsto A \not\subseteq \emptyset \\
\lnot \text{nonempty}(A)    &\leadsto A \subseteq \emptyset \\
A \subseteq B \cup (C \backslash D)
    &\leadsto A \subseteq B \cup C \land A \cap D \subseteq B \\
A \subseteq B \backslash C
    &\leadsto A \subseteq B \land A \cap C \subseteq \emptyset \\
A \cap (B \backslash D) \subseteq C
    &\leadsto A \cap B \subseteq C \cup D \\
A \backslash B \subseteq C
    &\leadsto A \subseteq B \cup C \\
A \cap (B \cup D) \subseteq C
    & \leadsto A \cap B \subseteq C \land A \cap D \subseteq C \\
A \cup B \subseteq C
    & \leadsto A \subseteq C \land B \subseteq C \\
A \subseteq B \cup (C \cap D)
    & \leadsto A \subseteq B \cup C \land A \subseteq B \cup D \\
A \subseteq B \cap C
    & \leadsto A \subseteq B \land A \subseteq C \\
A \not\subseteq B \cup (C \backslash D)
    &\leadsto A \not\subseteq B \cup C \lor A \cap D \not\subseteq B \\
A \not\subseteq B \backslash C
    &\leadsto A \not\subseteq B \lor A \cap C \not\subseteq \emptyset \\
A \cap (B \backslash D) \not\subseteq C
    &\leadsto A \cap B \not\subseteq C \cup D \\
A \backslash B \not\subseteq C
    &\leadsto A \not\subseteq B \cup C \\
A \cap (B \cup D) \not\subseteq C
    & \leadsto A \cap B \not\subseteq C \lor A \cap D \not\subseteq C \\
A \cup B \not\subseteq C
    & \leadsto A \not\subseteq C \lor B \not\subseteq C \\
A \not\subseteq B \cup (C \cap D)
    & \leadsto A \not\subseteq B \cup C \lor A \not\subseteq B \cup D \\
A \not\subseteq B \cap C
    & \leadsto A \not\subseteq B \lor A \not\subseteq C \\
A \cap \emptyset
    & \leadsto \emptyset \\
A \cup \emptyset
    & \leadsto A \\
\emptyset \subseteq B
    & \leadsto \top \\
\emptyset \not\subseteq B
    & \leadsto \bot \\
\{ s \} \subseteq \emptyset
    & \leadsto \bot \\
\{ s \} \not\subseteq \emptyset
    & \leadsto \top \\
\end{align*}
\caption{Skolemization and CNF rules}
\label{fig:skolem}
\end{center}
%}}}
\end{figure}

%}}}


\section{Inference Rules}
%{{{
We mix \emph{The Strive-based Subset Prover} and
\emph{Rewrite Techniques for Transitive Relations}.
We will use monotonicity properties $A \subseteq B \Rightarrow A \cap C
\subseteq B \cap C$ and $A \subseteq B \Rightarrow A \cup C \subseteq B \cup C$
a lot. The rules are exposed in Figure~\ref{fig:rules}.

Let us develop the chaining inferences. Say we have $A \subseteq t\cup B$
and $t \cap B' \subseteq C'$, then we can deduce intermediate literals
$A \cap B' \subseteq (t \cup B) \cap B'$ and
$(t\cap B')\cup B \subseteq C' \cup B$. Then, we use the fact that
$(t \cup B) \cap B' = (t\cap B') \cup (B\cap B')$, and that
$B \cap B' \subseteq B$, so $(t\cap B') \cup (B\cap B') \subseteq (t \cap B')\cup B$.
Therefore by transitivity of $\subseteq$, we obtain
$A\cap B' \subseteq C' \cup B$.

Conversely, the negative chaining rules use the contrapositive of transitivity,
that is, $A \subseteq B \land A \not\subseteq C \Rightarrow B \not\subseteq C$
(otherwise we would have $A \subseteq B \subseteq C$ and therefore $A \subseteq C$).
Therefore, from $A \cap t \subseteq B$ and $A' \cap t \not\subseteq B'$ we
deduce $A\cap A'\cap t \subseteq B\cap A'$
and $A\cap A'\cap t \not\subseteq B'\cap A$ by monotonicity. Therefore
$B\cap A' \not\subseteq B'\cap A$

\begin{figure}[htp]
%{{{
\begin{center}

\begin{prooftree}
    \AxiomC{$ K\lor (A \subseteq t \cup B) $}
    \AxiomC{$ K'\lor (t \cap B' \subseteq C') $}
    \RightLabel{positive chaining}
    \BinaryInfC{$ K \lor K' \lor (A \cap B' \subseteq B \cup C') $}
\end{prooftree}

\caption{Inference Rules}
\label{fig:rules}
\end{center}
%}}}
\end{figure}

%}}}


\end{document}
