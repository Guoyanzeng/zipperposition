\documentclass{article}

%{{{ prelude
\usepackage{bussproofs}
\usepackage[english]{babel}
\usepackage{amsmath}
\usepackage{amsfonts}
\usepackage{amssymb}
\usepackage{amsthm}
\usepackage{paralist}
\usepackage[colorlinks=false,pdfborder={0 0 0},
pdftitle={superposition and chaining with sets},pdfauthor={Simon Cruanes}]{hyperref}

\newcommand{\set}[1]{\ensuremath{\text{set}({#1})}}
\newcommand{\total}[1]{\ensuremath{\text{total}({#1})}}
\newcommand{\compl}[1]{\ensuremath{\overline{#1}}}
\newcommand{\normalize}[1]{\ensuremath{\text{norm}(#1)}}

\title{Superposition and Chaining with Sets}
%}}}

\begin{document}

\maketitle

\section{Basics}
%{{{

\subsection{Definitions}
%{{{
Literals are of the form $\bigcap_i a_i \subseteq \bigcup_j b_j$
or $\bigcap_i a_i \not\subseteq \bigcup_j b_j$ where
the $a_i$ and $b_j$ are terms of type set. 
The terms of type $\set{\alpha}$ are of the form:

\begin{itemize}
    \item $\{ t \}$ where $t : \alpha$;
    \item $\compl{t}$ where $t : \set{\alpha}$, with $\compl{\compl{t}} \equiv t$.
        This denotes the complement of $t$, which is well defined because
        the type $\set{\alpha}$ contains an element which is a set of all
        elements of type $\alpha$. Those can always be removed from literals
        as we will see later;
    \item $ f(t_1,\dots,t_n) $ where $f$ is a function of codomain $\set{\alpha}$;
    \item $ x $ where $x:\set{\alpha}$ is a variable.
\end{itemize}
%}}}

\subsection{Properties}\label{sec:properties}
%{{{
Let us start with some basic properties we are going to use.

\begin{itemize}
\item $A \cap B \subseteq C$ implies $A \subseteq \compl{B} \cup C$. All
    $x \in A\cap B$ satisfy $x \in A \land x \in C \land x \not\in \compl{B}$,
    therefore any $x\in A$ either satisfies $x \not\in B$ or $x \in B \land x \in C$.
    Which means precisely $A \subseteq \compl{B} \cup C$.
\item Similarly, $A \subseteq B \cup C \Rightarrow A \cap \compl{B} \subseteq C$.
\item $A \not\subseteq B \cup C$ implies $A \cap \compl{B} \not\subseteq C$.
    Proof:
    $\exists x, x \in A \land x \not\in (B \cup C)$, hence
    $\exists x, x \in A \land x \not\in B \land x \not\in C$,
    $\exists x, x \in A \land x \in \compl{B} \land x \not\in C$,
    $A \cap \compl{B} \not\subseteq C$.
\item $A \cap B \not\subseteq C$ implies $A \not\subseteq \compl{B}\cup C$.
    Indeed, $\exists x, x \in A \land x \in B \land x \not\in C$, so
    $\exists x, x \in A \land x \not\in \compl{B} \land x \not\in C$,
    so $\exists x, x\in A \land x \not\in (\compl{B} \cup C)$, and
    therefore $A \not\subseteq \compl{B} \cup C$.
\end{itemize}


Clause normalization rules are as follows, using distributivity of $\cap$
over $\cup$ and conversely:
\begin{itemize}
\item a clause $K \lor (A \subseteq B \cap C)$ will be normalized into the
    set of clauses
    $\{ K \lor (A \subseteq B),  K \lor (A \subseteq C) \}$.
\item a clause $K \lor (A \cup B \subseteq C)$ will become
    $\{ K \lor (A \subseteq C) , K \lor (B \subseteq C) \}$.
\item a clause $K \lor (A \cup B \not\subseteq C)$ will become
    $K \lor (A \not\subseteq C) \lor (B \not\subseteq C)$ (because the witness
    of non-inclusion is either in $A$ or in $B$).
\item a clause $K \lor (A \not\subseteq B \cap C)$ will become
    $K \lor (A \not\subseteq B) \lor (A \not\subseteq C)$ because
    the witness $x \in A$ can be either outside of $B$ or outside of $C$.
\end{itemize}
%}}}

\subsection{Powerset}
%{{{
\noindent{}We can encode the powerset operator $\mathbb{P}$ as follows, using a fresh
variable $s:\set{\alpha}$ and a fresh skolem
symbol $k:\set{\alpha}$ where $\mathbb{P}: \set{\set{\alpha}}$. Similar
techniques should be usable to eliminate infinite
union (of type $\set{\set{\alpha}} \rightarrow \set{\alpha}$) and other operators.

\begin{itemize}
\item If $\mathbb{P}(t) \cap A \subseteq B$,
    it means $\forall s, s\in \mathbb{P}(t) \cap A \Rightarrow s \in B$;
    $s \in \mathbb{P}(t) \cap A$ is synonymous to $s \subseteq t \land
    \{ s \} \subseteq A$.
\item Similarly, $A \subseteq \mathbb{P}(t) \cup B$
    means $\forall s, s\in A \Rightarrow s \in \mathbb{P}(t) \lor s\in B$,
    hence $\forall s, \{s \}\subseteq A \Rightarrow s \subseteq t \lor \{ s \} \subseteq B$.
\item $A \cap \mathbb{P}(t) \not\subseteq B$ means
    $\exists s, s \in A \land s \subseteq t \land s \not\in B$.
    Therefore $\{k\} \subseteq A \land k \subseteq t \land \{k\} \not\subseteq B$.
\item $A \not\subseteq \mathbb{P}(t) \cup B$ becomes
    $\{ k\} \subseteq A \land \{k\} \not\subseteq B \land k \not\subseteq t$.
\end{itemize}

When $\mathbb{P}$ occurs in a clause, the rules to eliminate it are shown
in Figure~\ref{fig:powerset}.

\begin{figure}[htp]
%{{{
\begin{prooftree}
\AxiomC{$ C \lor \mathbb{P}(t) \cap A \subseteq B$}
\doubleLine
\UnaryInfC{$C \lor s \not\subseteq t \lor \{ s \} \not\subseteq A \lor \{ s \} \subseteq B$}
\end{prooftree}

\begin{prooftree}
\AxiomC{$ C \lor \mathbb{P}(t) \cap A \not\subseteq B$}
\doubleLine
\UnaryInfC{$C \lor \{k\} \subseteq A
    \qquad C \lor k \subseteq t
    \qquad C \lor \{k\} \not\subseteq B $}
\end{prooftree}

\begin{prooftree}
\AxiomC{$ C \lor A \subseteq \mathbb{P}(t) \cup B$}
\doubleLine
\UnaryInfC{$C \lor \{ s \} \not\subseteq A \lor s \subseteq t \lor \{ s \} \subseteq B$}
\end{prooftree}

\begin{prooftree}
\AxiomC{$ C \lor A \not\subseteq \mathbb{P}(t) \cup B$}
\doubleLine
\UnaryInfC{$C \lor \{ k \} \subseteq A
    \qquad C \lor \{k\} \not\subseteq B
    \qquad C \lor k \not\subseteq t $}
\end{prooftree}

\caption{Elimination of $\mathbb{P}$}
\label{fig:powerset}
\end{figure}
%}}}

%}}}
%}}}

\section{Skolemisation and CNF}
%{{{
The numerous set operators must be translated to the rigid clause format
we defined above; Figure~\ref{fig:skolem} describes the translation process
that must take place after NNF and before Skolemization.
In the rules, $A$ and $B$ have type $\set{\alpha}$, $s$, $t$, $u$, $v$ are terms
of type $\alpha$, and $x:\alpha$ is a
variable. We de-sugar $A \backslash B$ into $A \cap \compl{B}$.
Then, usual CNF transformation, with skolemization
and distribution of $\land$ over $\lor$, can proceed as usual.

\begin{figure}[htbp]
%{{{
\begin{center}
\begin{align*}
% basics
A \not\subset B &\leadsto (B \subseteq A \lor A \not\subseteq B) \\
A \not= B       &\leadsto (A \not\subseteq B \lor B \not\subseteq A) \\
A \subset B     &\leadsto (A \subseteq B \land B \not\subseteq A) \\
A = B           &\leadsto (A \subseteq B \land B \subseteq A) \\
s \in A         &\leadsto \{ s \} \subseteq A \\
s \not\in A     &\leadsto \{ s \} \not\subseteq A \\
\text{nonempty}(A)          &\leadsto A \not\subseteq \emptyset \\
\lnot \text{nonempty}(A)    &\leadsto A \subseteq \emptyset \\
% difference
A \backslash B
    & \leadsto A \cap \compl{B} \\
% subseteq
A \cap (B \cup D) \subseteq C
    & \leadsto A \cap B \subseteq C \land A \cap D \subseteq C \\
A \cup B \subseteq C
    & \leadsto A \subseteq C \land B \subseteq C \\
A \subseteq B \cup (C \cap D)
    & \leadsto A \subseteq B \cup C \land A \subseteq B \cup D \\
A \subseteq B \cap C
    & \leadsto A \subseteq B \land A \subseteq C \\
% not subseteq (FIXME)
A \cup B \not\subseteq C
    & \leadsto A \not\subseteq C \lor B \not\subseteq C \\
A \cap (B \cup C) \not\subseteq D
    & \leadsto A \cap B \not\subseteq D \lor A \cap C \not\subseteq D \\
A \not\subseteq B \cap C
    & \leadsto A \not\subseteq B \lor A \not\subseteq C \\
A \not\subseteq B \cup (C \cap D)
    & \leadsto A \not\subseteq B \cup C \lor A \not\subseteq B \cup D \\
% empty set, singleton
A \cap \emptyset
    & \leadsto \emptyset \\
A \cup \emptyset
    & \leadsto A \\
\emptyset \subseteq B
    & \leadsto \top \\
\emptyset \not\subseteq B
    & \leadsto \bot \\
\{ s \} \subseteq \emptyset
    & \leadsto \bot \\
\{ s \} \not\subseteq \emptyset
    & \leadsto \top \\
\end{align*}
\caption{Skolemization and CNF rules}
\label{fig:skolem}
\end{center}
%}}}
\end{figure}

%}}}

\section{Inference Rules}
%{{{
We mix \emph{The Strive-based Subset Prover} and
\emph{Rewrite Techniques for Transitive Relations} papers for our
inference rules.
We will use basic properties detailed in section~\ref{sec:properties}.
We will use capitalized
letter such as $A_\cap$ to denote $\bigcap_i a_i$, and $A_\cup$ to denote
$\bigcup_i a_i$. Note that $\compl{A_\cap}$ is a union and
$\compl{A_\cup}$ is an intersection.

All the inference operate on ground clauses, but can easily be lifted to
first-order terms by first unifying terms. Since $\cap$ and $\cup$ are
idempotent, several terms on one side of a literal can be unified at once
without trouble (unlike, for instance, unification of terms in arithmetic
sums or surrounded with AC operators). For instance positive chaining
for first-order clauses is (assuming the sets $I$ and $J$ are non empty,
none of the $t_i, t'_j$ are variables
and the two clauses do not share variables):

\begin{prooftree}
    \AxiomC{$ K \lor A_\cap \cap \bigcap_{i\in I} t_i \subseteq B_\cup $}
    \AxiomC{$ K' \lor A'_\cap \subseteq B'_\cup \cup \bigcup_{j\in J} t'_j $}
    \RightLabel{$\forall i,j,~ t_i\sigma = t'_j\sigma$}
    \BinaryInfC{$ (K \lor K' \lor A_\cap \cap A'_\cap
        \subseteq B_\cup \cup B'_\cup)\sigma $}
\end{prooftree}

Let us explain and justify the inferences. We will note $\normalize{C}$ for
the set of clauses that result from normalizing $C$ by distributing $\cap$
and $\cup$.

First, the positive chaining inference. Say we have $A_\cap \subseteq t\cup B_\cup$
and $t \cap A'_\cap \subseteq B'_\cup$. Then this is equivalent
to $A_\cap \cap \compl{B_\cup} \subseteq t$ and
$t \subseteq \compl{A'_\cap}\cup B'_\cup$; by transitivity of
$\subseteq$ we get
$A_\cap \cap \compl{B_\cup} \subseteq B'_\cup \cup \compl{A'_\cap}$, and by moving
the complements around again we obtain the result.

Conversely, the negative chaining rules use the contrapositive of transitivity,
that is, $A \subseteq B \land A \not\subseteq C \Rightarrow B \not\subseteq C$
(otherwise we would have $A \subseteq B \subseteq C$ and therefore $A \subseteq
C$).
Therefore, from $A_\cap \cap t \subseteq B_\cup$ and
$A'_\cap \cap t \not\subseteq B'_\cup$ we deduce
$t \subseteq B_\cup \cup \compl{A_\cap}$ and
$t \not\subseteq B'_\cup \cup \compl{A'_\cap}$. Using the transitivity,
we obtain $B_\cup \cup \compl{A_\cap} \not\subseteq B'_\cup \cup \compl{A'_\cap}$.

A similar reasoning applies when $t$ occurs at the right-hand side of
$\subseteq$ and $\not\subseteq$, using the transitivity rule
$A \subseteq B \land C \not\subseteq B \Rightarrow C \not\subseteq A$. We
obtain, from $A_\cap \subseteq t\cup B_\cup$ and
$A'_\cap \not\subseteq t\cup B'_\cup$, the result
$A'_\cap \cap \compl{B'_\cup} \not\subseteq A_\cap \cap \compl{B_\cup}$.

Factoring rules are used to \emph{merge} together two literals whose maximal
terms are the same. The basic idea is that if $A \subseteq B
\lor A \subseteq C$, then if we assume $B\subseteq C$ we can deduce
$A \subseteq B \Rightarrow A \subseteq C$. Therefore $A\subseteq C$ is true
in both cases and we can merge together the two literals provided $B\subseteq C$.
Say we have
$t \cap A_\cap \subseteq B_\cup \lor t \cap A'_\cap \subseteq B'_\cup$; we
can transform this into
$t \subseteq \compl{A_\cap} \cup B_\cup \lor t \subseteq \compl{A'_\cap} \cup B'_\cup$.
Then if $\compl{A_\cap} \cup B_\cup \subseteq \compl{A'_\cap} \cup B'_\cup$
the second literal is necessarily true, hence the conclusion:
$\normalize{(\compl{A_\cap} \cap A'_\cap \not\subseteq \compl{B_\cup} \cup B'_\cup) 
    \lor (t \cap A'_\cap \subseteq B'_\cup)}$ (eliminating complements
on the fly).

\begin{figure}[htp]
%{{{
\begin{center}

\begin{prooftree}
    \AxiomC{$ K\lor (A_\cap \subseteq t \cup B_\cup) $}
    \AxiomC{$ K'\lor (t \cap A'_\cap \subseteq B'_\cup) $}
    \RightLabel{positive chaining}
    \BinaryInfC{$ K \lor K' \lor (A_\cap \cap A'_\cap \subseteq B_\cup \cup B'_\cup) $}
\end{prooftree}

\begin{prooftree}
    \AxiomC{$ K\lor (A_\cap \cap t \subseteq B_\cup) $}
    \AxiomC{$ K'\lor (A'_\cap \cap t \not\subseteq B'_\cup) $}
    \RightLabel{negative chaining left}
    \BinaryInfC{$ \normalize{K \lor K' \lor
        (B_\cup \cup \compl{A_\cap} \not\subseteq \compl{A'_\cap} \cup B'_\cup)}
    $}
    \UnaryInfC{$
        K \lor K'
        \lor \bigvee_{s \in B_\cup} (A'_\cap \cap s \subseteq B'_\cup)
        \lor \bigvee_{s \in A_\cap} (A'_\cap \subseteq s \cup B'_\cup)
    $}
\end{prooftree}

\begin{prooftree}
    \AxiomC{$ K\lor (A_\cap \subseteq B_\cup \cup t) $}
    \AxiomC{$ K'\lor (A'_\cap \not\subseteq B'_\cup \cup t) $}
    \RightLabel{negative chaining right}
    \BinaryInfC{$ \normalize{K \lor K' \lor
        (A'_\cap \cap \compl{B'_\cup} \not\subseteq A_\cap \cap \compl{B_\cup})}
    $}
    \UnaryInfC{$
        K \lor K'
        \lor \bigvee_{s \in A_\cap} (A'_\cap \subseteq s \cup B'_\cup)
        \lor \bigvee_{s \in B_\cup} (A'_\cap \cap s \subseteq B'_\cup)
    $}
\end{prooftree}

\begin{prooftree}
    \AxiomC{$ K
        \lor (t \cap A_\cap \subseteq B_\cup)
        \lor (t \cap A'_\cap \subseteq B'_\cup) $}
    \RightLabel{factoring left}
    \UnaryInfC{$ \normalize{K \lor
        (\compl{A_\cap} \cap A'_\cap \not\subseteq \compl{B_\cup} \cup B'_\cup) 
        \lor (t \cap A'_\cap \subseteq B'_\cup)}$}
    \UnaryInfC{$
        K
        \lor (t \cap A'_\cap \subseteq B'_\cup)
        \lor \bigvee_{u \in A_\cap}
            \bigvee_{v \in B_\cup}
                (v \cap A'_\cap \not\subseteq u \cup B'_\cup)
    $}
\end{prooftree}

\mbox{
    \AxiomC{$ K \lor \{ t \} \subseteq \emptyset $}
    \doubleLine
    \UnaryInfC{$ K $}
    \DisplayProof

    and

    \AxiomC{$ K \lor \{ t \} \not\subseteq \emptyset $}
    \RightLabel{emptyset}
    \doubleLine
    \UnaryInfC{$\top$}
    \DisplayProof
} \\[10pt]


\mbox{
    \AxiomC{$ K \lor t \cap A_\cap \subseteq t\cup B_\cup $}
    \doubleLine
    \UnaryInfC{$\top $}
    \DisplayProof

    and

    \AxiomC{$ K \lor t \cap A_\cap \not\subseteq t\cup B_\cup $}
    \RightLabel{reflexivity}
    \doubleLine
    \UnaryInfC{$K$}
    \DisplayProof
} \\[10pt]



\caption{Inference Rules}
\label{fig:rules}
\end{center}
%}}}
\end{figure}

%}}}

\section{Other Considerations}
%{{{
\subsection{Handling Equality}
%{{{
As we saw in Figure~\ref{fig:rules}, the calculus doesn't deal directly
with (extensional) equality between sets. Instead, $ a = b$ is
translated into $a \subseteq b \land b \subseteq a$. However, replacing
equals by equals in set literals is still possible.
We assume $u=v$ was translated into $u \subseteq v\land v\subseteq u$,
and show that we can replace $u$ with $v$ in set contexts:

\begin{prooftree}
    \AxiomC{$ K \lor u \cap A_\cap \subseteq B_\cup $}
    \AxiomC{$ K' \lor v \subseteq u $ }
    \BinaryInfC{$ K \lor K' \lor v \cap A_\cap \subseteq B_\cup $}
\end{prooftree}

and

\begin{prooftree}
    \AxiomC{$ K \lor A_\cap \subseteq u \cup B_\cup $}
    \AxiomC{$ K' \lor u \subseteq v $ }
    \BinaryInfC{$ K \lor K' \lor A_\cap \subseteq v \cup B_\cup $}
\end{prooftree}
%}}}

\subsection{Purification}
%{{{
More difficult is the treatment of set-typed terms that occur under function
symbols (or predicate symbols), for instance $p(s)$ when $s:\set{\alpha}$.
The usual solution to this issue is
to replace $C[p(s)]$ (where $C[.]$ is a clause context) with
$C[p(x)] \lor x \not= s$, that is, replacing $s$ with a fresh variable
$x\not\in \text{vars}(C[p(s)])$ and force $x=s$. We then obtain
$C[p(x)] \lor x\not\subseteq s \lor s\not\subseteq x$.
%}}}


\subsection{Subsumption}
%{{{
It is often useful to have a \emph{subsumption relation}, to prune the search
space by removing clauses that are implied by smaller, simpler clauses. This
relation on clauses is deduced from a subsumption relation on literals (most
often, just matching). Here, we can use the subsumption relation $\models_s$ defined
by

\begin{align*}
(A_\cap \subseteq B_\cup) &\models_s
    (A_\cap\sigma \cap A'_\cap \subseteq B_\cup\sigma \cup B'_\cup) \\
(A_\cap \cap A'_\cap \not\subseteq B_\cup \cup B'_\cup) &\models_s
    (A_\cap\sigma\not\subseteq B_\cup\sigma) \\
\end{align*}

This relation, on positive literals, corresponds to instantiation and
weakening on the left and right sides of a sequent. The negative counterpart
is very interesting as it departs from the usual subsumption rules, the
more specialized (instantiated) literal being on the left.
%}}}

%}}}  section



\end{document}
